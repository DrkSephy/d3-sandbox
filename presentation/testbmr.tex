\documentclass[red]{beamer}
% Class options include: notes, notesonly, handout, trans,
%                        hidesubsections, shadesubsections,
%                        inrow, blue, red, grey, brown


% Theme for beamer presentation.
\usepackage{beamerthemetree} 
% Other themes include: beamerthemebars, beamerthemelined, 
%                       beamerthemetree, beamerthemetreebars  
\usepackage{listings}

\title{D3: Data Driven Documents}    
\author{David Leonard}                 
\institute{City College of New York}      
\date{\today}                   

\begin{document}

\lstdefinelanguage{JavaScript}{
  keywords={typeof, new, true, false, catch, function, return, null, catch, switch, var, if, in, while, do, else, case, break, this},
  keywordstyle=\color{blue}\bfseries,
  ndkeywords={class, export, boolean, throw, implements, import, this},
  ndkeywordstyle=\color{darkgray}\bfseries,
  identifierstyle=\color{black},
  sensitive=false,
  comment=[l]{//},
  morecomment=[s]{/*}{*/},
  commentstyle=\color{purple}\ttfamily,
  stringstyle=\color{red}\ttfamily,
  morestring=[b]',
  morestring=[b]"
}

\lstset{
   language=JavaScript,
   extendedchars=true,
   basicstyle=\scriptsize\ttfamily,
   showstringspaces=false,
   showspaces=false,
   tabsize=2,
   breaklines=true,
   showtabs=false,
   captionpos=b
}

% Object code listing
\defverbatim[colored]\lstI{
	\begin{lstlisting}
		var person = {
			name: David,
			age: 23,
			major: Computer Science
		};
		
		person.name; // David
		person.age; // 23
		person.major; // Computer Science 
	\end{lstlisting}
}

\defverbatim[colored]\lstll{
    \begin{lstlisting}
    var paragraphs = document.getElementsByTagName("p");
    for (var i = 0; i < paragraphs.length; i++) {
        var paragraph = paragraphs.item(i);
        paragraph.style.setProperty("color", "white", null);
    } 
    \end{lstlisting}
}

% Creates title page of slide show using above information
\begin{frame}
  \titlepage
\end{frame}

\section[Outline]{}

\section{Introduction to D3}

\begin{frame}
    \frametitle{What is D3?}
    D3 is a JavaScript library which allows you to bind arbitrary datasets to 
    the Document Object Model (DOM) and exposes methods for performing data-driven operations to the document.  
\end{frame}


\subsection{Power of D3}

\begin{frame}
  \frametitle{Loading Data}   % Insert frame title between curly braces
  D3 provides numerous ways to introduce data to your objects. 
  \begin{itemize}
  \item<1-> Array of values
  \item<2-> Tab-seperated values
  \item<3-> Comma-seperated values
  \item<4-> JSON
  \end{itemize}
\end{frame}

\begin{frame}
	\frametitle{Selectors}
    Using JavaScripts native API to manipulate the DOM can be tiresome. Libraries such as jQuery provide abstractions to make this easier, and D3 exposes its own methods for working with the DOM. 

\end{frame}

\begin{frame}
 	\frametitle{JavaScript Objects}
		\lstI
\end{frame}

\section{Understanding ImpactJS Game Loop}

\end{document}
